% Options for packages loaded elsewhere
\PassOptionsToPackage{unicode}{hyperref}
\PassOptionsToPackage{hyphens}{url}
%
\documentclass[
]{article}
\title{FLS301\_2021}
\author{}
\date{\vspace{-2.5em}}

\usepackage{amsmath,amssymb}
\usepackage{lmodern}
\usepackage{iftex}
\ifPDFTeX
  \usepackage[T1]{fontenc}
  \usepackage[utf8]{inputenc}
  \usepackage{textcomp} % provide euro and other symbols
\else % if luatex or xetex
  \usepackage{unicode-math}
  \defaultfontfeatures{Scale=MatchLowercase}
  \defaultfontfeatures[\rmfamily]{Ligatures=TeX,Scale=1}
\fi
% Use upquote if available, for straight quotes in verbatim environments
\IfFileExists{upquote.sty}{\usepackage{upquote}}{}
\IfFileExists{microtype.sty}{% use microtype if available
  \usepackage[]{microtype}
  \UseMicrotypeSet[protrusion]{basicmath} % disable protrusion for tt fonts
}{}
\makeatletter
\@ifundefined{KOMAClassName}{% if non-KOMA class
  \IfFileExists{parskip.sty}{%
    \usepackage{parskip}
  }{% else
    \setlength{\parindent}{0pt}
    \setlength{\parskip}{6pt plus 2pt minus 1pt}}
}{% if KOMA class
  \KOMAoptions{parskip=half}}
\makeatother
\usepackage{xcolor}
\IfFileExists{xurl.sty}{\usepackage{xurl}}{} % add URL line breaks if available
\IfFileExists{bookmark.sty}{\usepackage{bookmark}}{\usepackage{hyperref}}
\hypersetup{
  pdftitle={FLS301\_2021},
  hidelinks,
  pdfcreator={LaTeX via pandoc}}
\urlstyle{same} % disable monospaced font for URLs
\usepackage[margin=1in]{geometry}
\usepackage{color}
\usepackage{fancyvrb}
\newcommand{\VerbBar}{|}
\newcommand{\VERB}{\Verb[commandchars=\\\{\}]}
\DefineVerbatimEnvironment{Highlighting}{Verbatim}{commandchars=\\\{\}}
% Add ',fontsize=\small' for more characters per line
\usepackage{framed}
\definecolor{shadecolor}{RGB}{248,248,248}
\newenvironment{Shaded}{\begin{snugshade}}{\end{snugshade}}
\newcommand{\AlertTok}[1]{\textcolor[rgb]{0.94,0.16,0.16}{#1}}
\newcommand{\AnnotationTok}[1]{\textcolor[rgb]{0.56,0.35,0.01}{\textbf{\textit{#1}}}}
\newcommand{\AttributeTok}[1]{\textcolor[rgb]{0.77,0.63,0.00}{#1}}
\newcommand{\BaseNTok}[1]{\textcolor[rgb]{0.00,0.00,0.81}{#1}}
\newcommand{\BuiltInTok}[1]{#1}
\newcommand{\CharTok}[1]{\textcolor[rgb]{0.31,0.60,0.02}{#1}}
\newcommand{\CommentTok}[1]{\textcolor[rgb]{0.56,0.35,0.01}{\textit{#1}}}
\newcommand{\CommentVarTok}[1]{\textcolor[rgb]{0.56,0.35,0.01}{\textbf{\textit{#1}}}}
\newcommand{\ConstantTok}[1]{\textcolor[rgb]{0.00,0.00,0.00}{#1}}
\newcommand{\ControlFlowTok}[1]{\textcolor[rgb]{0.13,0.29,0.53}{\textbf{#1}}}
\newcommand{\DataTypeTok}[1]{\textcolor[rgb]{0.13,0.29,0.53}{#1}}
\newcommand{\DecValTok}[1]{\textcolor[rgb]{0.00,0.00,0.81}{#1}}
\newcommand{\DocumentationTok}[1]{\textcolor[rgb]{0.56,0.35,0.01}{\textbf{\textit{#1}}}}
\newcommand{\ErrorTok}[1]{\textcolor[rgb]{0.64,0.00,0.00}{\textbf{#1}}}
\newcommand{\ExtensionTok}[1]{#1}
\newcommand{\FloatTok}[1]{\textcolor[rgb]{0.00,0.00,0.81}{#1}}
\newcommand{\FunctionTok}[1]{\textcolor[rgb]{0.00,0.00,0.00}{#1}}
\newcommand{\ImportTok}[1]{#1}
\newcommand{\InformationTok}[1]{\textcolor[rgb]{0.56,0.35,0.01}{\textbf{\textit{#1}}}}
\newcommand{\KeywordTok}[1]{\textcolor[rgb]{0.13,0.29,0.53}{\textbf{#1}}}
\newcommand{\NormalTok}[1]{#1}
\newcommand{\OperatorTok}[1]{\textcolor[rgb]{0.81,0.36,0.00}{\textbf{#1}}}
\newcommand{\OtherTok}[1]{\textcolor[rgb]{0.56,0.35,0.01}{#1}}
\newcommand{\PreprocessorTok}[1]{\textcolor[rgb]{0.56,0.35,0.01}{\textit{#1}}}
\newcommand{\RegionMarkerTok}[1]{#1}
\newcommand{\SpecialCharTok}[1]{\textcolor[rgb]{0.00,0.00,0.00}{#1}}
\newcommand{\SpecialStringTok}[1]{\textcolor[rgb]{0.31,0.60,0.02}{#1}}
\newcommand{\StringTok}[1]{\textcolor[rgb]{0.31,0.60,0.02}{#1}}
\newcommand{\VariableTok}[1]{\textcolor[rgb]{0.00,0.00,0.00}{#1}}
\newcommand{\VerbatimStringTok}[1]{\textcolor[rgb]{0.31,0.60,0.02}{#1}}
\newcommand{\WarningTok}[1]{\textcolor[rgb]{0.56,0.35,0.01}{\textbf{\textit{#1}}}}
\usepackage{graphicx}
\makeatletter
\def\maxwidth{\ifdim\Gin@nat@width>\linewidth\linewidth\else\Gin@nat@width\fi}
\def\maxheight{\ifdim\Gin@nat@height>\textheight\textheight\else\Gin@nat@height\fi}
\makeatother
% Scale images if necessary, so that they will not overflow the page
% margins by default, and it is still possible to overwrite the defaults
% using explicit options in \includegraphics[width, height, ...]{}
\setkeys{Gin}{width=\maxwidth,height=\maxheight,keepaspectratio}
% Set default figure placement to htbp
\makeatletter
\def\fps@figure{htbp}
\makeatother
\setlength{\emergencystretch}{3em} % prevent overfull lines
\providecommand{\tightlist}{%
  \setlength{\itemsep}{0pt}\setlength{\parskip}{0pt}}
\setcounter{secnumdepth}{-\maxdimen} % remove section numbering
\ifLuaTeX
  \usepackage{selnolig}  % disable illegal ligatures
\fi

\begin{document}
\maketitle

\hypertarget{inferential-stats-with-r}{%
\section{Inferential Stats with R}\label{inferential-stats-with-r}}

Empowerment for Local People Foundation Lagos , Dec 8-10

sugar \textless- 1+2

salt \textless- 8-2

5+(6-2)*5 sugar \textless- 1+2

\hypertarget{but-first-a-quick-look-at-working-directories}{%
\subsubsection{But first, a quick look at working
directories}\label{but-first-a-quick-look-at-working-directories}}

getwd()

setwd(``C:/Users/User/Dropbox/Bookkeeping'') \# This is how to set
working directory

setwd(``/Users/antheaalberto/Desktop/Stats I (TA)/Labs/Lab2'')

\hypertarget{copying-path-directly-is-the-easiest-way-to-do-this}{%
\subsubsection{Copying path directly is the easiest way to do
this}\label{copying-path-directly-is-the-easiest-way-to-do-this}}

\hypertarget{set-default-working-directory-tools-global-options}{%
\subsubsection{Set default working directory: Tools \textgreater{}
Global
Options}\label{set-default-working-directory-tools-global-options}}

\hypertarget{we-can-make-them-ourselves}{%
\subsubsection{We can make them
ourselves}\label{we-can-make-them-ourselves}}

\hypertarget{concecunate-which-is-c-allows-us-to-group-different-things-into-one-object}{%
\subsubsection{concecunate which is `c' allows us to group different
things into one
object}\label{concecunate-which-is-c-allows-us-to-group-different-things-into-one-object}}

Gender \textless- c(``Male'', ``Male'', ``Male'', ``Male'', ``Male'',
``Male'', ``Male'', ``Male'', ``Male'', ``Male'', ``FeMale'',``FeMale'',
``FeMale'',``FeMale'', ``FeMale'',
``FeMale'',``FeMale'',``FeMale'',``FeMale'',``FeMale'') weight
\textless- c(89, 75, 88, 75, 49, 89, 110, 120, 89, 75, 75, 76, 87, 110,
67, 76, 43, 55, 59, 60) income \textless- c(50000, 95000, 120000,
800000, 650000, 92000, 94000, 222000, 543000,75000, 63000, 40000, 99000,
450000, 180000, 190000, 96000, 780000, 150000, 342000) rating \textless-
c(5, 1, 2, 4, 9, 9, 8, 1, 9, 7, 5, 6, 6, 1, 1, 1, 3, 6, 9, 4) Marstatus
\textless- c(``Married'', ``Married'',``Single'', ``Single'',``Single'',
``Single'', ``Divorced'',``Single'', ``Married'',``Single'',
``Married'', ``Single'',``Single'', ``Divorced'',``Single'', ``Single'',
``Divorced'', ``Single'',``Married'', ``Divorced'') CityCentral
\textless-
c(``Yes'',``No'',``Yes'',``Yes'',``Yes'',``Yes'',``Yes'',``Yes'',``No'',``Yes'',
``No'',``No'',``Yes'',``Yes'',``No'',``Yes'',``No'',``Yes'',``No'',``No'')

officew \textless- as.data.frame(cbind(Gender, weight, income, rating,
Marstatus, CityCentral))

d \textless-
Day\_5\_Measuring\_Small\_Firm\_Behavior\_Survey\_latest\_version\_labels\_2019\_05\_13\_08\_01\_57

firstreg \textless-

pplwork\textless- d\$\texttt{How\ many\ people\ work\ with\ you?}

meanincome \textless-
mean(officew\(income) modeincome <- mode(officew\)income) modeincome
medianincome \textless-
median(officew\(income) varincome <- var(officew\)income)

officew\(income <- as.numeric(as.character(officew\)income)) sdincome
\textless- sd(officew\$income)

class(officew\$income)

\hypertarget{section}{%
\section{}\label{section}}

\hypertarget{types-of-data}{%
\section{TYPES OF DATA}\label{types-of-data}}

\#FACTOR \#CHARACTER (CH)/\#STRINGS \#NUMERICAL

\hypertarget{section-2}{%
\section{}\label{section-2}}

\hypertarget{understand-r-syntax}{%
\section{UNDERSTAND R SYNTAX}\label{understand-r-syntax}}

\#Vector \#Arguments \#Functions \#Data Frames \#Rows, Columns \#Value
\#Method \#Properties \#Matrices \#List \#Loops \#Packages \#Working
Directory

\hypertarget{first-we-are-taking-2-objects-into-a-column-and-we-telling-r}{%
\subsubsection{First we are taking 2 objects into a column and we
telling
R}\label{first-we-are-taking-2-objects-into-a-column-and-we-telling-r}}

\hypertarget{to-use-cbind-to-take-these-different-columns-and-see-them-as-one}{%
\subsubsection{to use cbind to take these different columns and see them
as
one}\label{to-use-cbind-to-take-these-different-columns-and-see-them-as-one}}

\hypertarget{a-data-frame-is-an-object-in-r}{%
\subsubsection{a data frame is an object in
R}\label{a-data-frame-is-an-object-in-r}}

\hypertarget{factor-is-another-way-of-calling-categorical-variable.-the-as.data.frame}{%
\subsubsection{factor is another way of calling categorical variable.
The
as.data.frame}\label{factor-is-another-way-of-calling-categorical-variable.-the-as.data.frame}}

\hypertarget{changes-the-factor-categorical-into-a-data-frame-without-necessarily}{%
\subsubsection{changes the factor (categorical) into a data frame
without
necessarily}\label{changes-the-factor-categorical-into-a-data-frame-without-necessarily}}

\hypertarget{changing-the-class}{%
\subsubsection{changing the class}\label{changing-the-class}}

\hypertarget{the-c-only-works-if-the-number-of-rows-in-each-variable-is-the-same}{%
\subsubsection{the `c' only works if the number of rows in each variable
is the
same}\label{the-c-only-works-if-the-number-of-rows-in-each-variable-is-the-same}}

head(officew, n = 5) \#\#\#\#\#\#\#\#\#\#\#\#\#\#\#\#\#\#\#\#\#\#\#\#
\#\#\# summary stats\#\#\#
\#\#\#\#\#\#\#\#\#\#\#\#\#\#\#\#\#\#\#\#\#\#\#\#

\hypertarget{let-us-use-rbind-rowbind-to-bind-the-rows-of-two-different-dataset}{%
\subsubsection{Let us use rbind (rowbind) to bind the rows of two
different
dataset}\label{let-us-use-rbind-rowbind-to-bind-the-rows-of-two-different-dataset}}

\hypertarget{together.-the-row-names-of-the-two-datasets-must-be-same-for-it-to-work}{%
\subsubsection{together. The row names of the two datasets must be same
for it to
work}\label{together.-the-row-names-of-the-two-datasets-must-be-same-for-it-to-work}}

Gender1 \textless- c(``Male'', ``Male'', ``Male'', ``Male'', ``Male'',
``Male'', ``Male'', ``Male'', ``Male'', ``Male'', ``Male'') weight1
\textless- c(89, 75, 88, 75, 49, 89, 110, 120, 89, NA, 75)

rating1 \textless- c(5, 1, 2, 4, 9, 9, 8, 1, 9,NA, 7, 5, 6, 6, 1, 1, 1,
3, 6, 9, 4)

officew\_men22\textless- as.data.frame(cbind(Gender1, weight1, rating1))
\#\#\# We are creating an

Gender1 \textless- c(``FeMale'',``FeMale'', ``FeMale'',``FeMale'',
``FeMale'', ``FeMale'',``FeMale'',``FeMale'',``FeMale'', ``FeMale'',
``FeMale'') weight1 \textless- c(75, 76, 87, 110, 67, 76, 43, NA, 55,
59, 60)

rating1 \textless- c(1, 5, 2, 4, 4, 9, 4, 1, 9, 5, 4, 6, 4, 1, 1, 4, 3,
6,NA, 9, 4)

officew\_women22\textless- as.data.frame(cbind(Gender1, weight1,
rating1)) \#\#\# We are creating an

officew\_full \textless- rbind(officew\_men22, officew\_women22)\# The
officew\_full data set \# is the same as the officew data set

\hypertarget{removing-nas}{%
\section{REMOVING NAs}\label{removing-nas}}

officew\_women22 \textless-
officew\_women22{[}!is.na(officew\_women22\(rating1)  &!is.na(officew_women22\)weight1),
{]}

\hypertarget{method-3}{%
\section{Method 3}\label{method-3}}

\hypertarget{using-the-subset-and-select-functions-to-select}{%
\section{Using the subset and select functions to
select}\label{using-the-subset-and-select-functions-to-select}}

\hypertarget{gender-variable-and-then-select-weight-as-the-required-row}{%
\section{Gender variable, and then select weight as the required
row}\label{gender-variable-and-then-select-weight-as-the-required-row}}

officew\_women1
\textless-(officew\(Gender="FeMale") officew_women2 <- subset(officew, Gender == "FeMale") officew_women3 <- subset(officew, Gender == "FeMale", select =  c("weight")) # male officew_men1 <-(officew\)Gender=``Male'')
officew\_men2 \textless- subset(officew, Gender == ``Male'')
officew\_men3 \textless- subset(officew, Gender == ``Male'', select =
c(``weight'')) \# For that we will be using several tools from the tidyr
and dplyr packages. \# Remember we already know some functions from this
packages: drop\_na, etc\ldots{}

summary(officew\_men3\(weight) summary(officew_women3\)weight) \#\#\#
standard deviation:
sd(officew\(weight, na.rm = T) ###for the sample size, we need to omit all missing values. the length(), which() and is.na() functions can help us: length(which(!is.na(officew\)weight)))

\hypertarget{include-3rd-and-5th-variable-income-and-marital-status}{%
\section{include 3rd and 5th variable (Income and marital
status)}\label{include-3rd-and-5th-variable-income-and-marital-status}}

officew\_3\_5 \textless- officew{[}c(3,5){]}

\hypertarget{exclude-3rd-and-5th-variable-income-and-marital-status}{%
\section{exclude 3rd and 5th variable (Income and marital
status)}\label{exclude-3rd-and-5th-variable-income-and-marital-status}}

officew\_1\_2\_4 \textless- officew{[}c(-3,-5){]} office\_1\_2\_4B
\textless- officew{[}c(1,2,4){]}

\hypertarget{include-1st-and-2nd-variablecolumn-and-4-and-the-4th-and-5th-observationrow}{%
\section{include 1st and 2nd variable(column), and 4 and the 4th and 5th
observation(row)}\label{include-1st-and-2nd-variablecolumn-and-4-and-the-4th-and-5th-observationrow}}

officew\_weight\_inc\_ \textless- officew{[}c(1:2),c(4:5){]}

office\_f\_hincome \textless- subset(officew, income
\textgreater=100000\textbar{} Gender \textless{} ``FeMale'',
select=c(1:5))

class(officew\$income)

\hypertarget{if-you-have-not-instaled-the-packages-you-have-to-install-them-first}{%
\section{If you have not instaled the packages, you have to install them
first!}\label{if-you-have-not-instaled-the-packages-you-have-to-install-them-first}}

\hypertarget{install.packagestidyr}{%
\section{install.packages(``tidyr'')}\label{install.packagestidyr}}

\hypertarget{install.packagesdplyr}{%
\section{install.packages(``dplyr'')}\label{install.packagesdplyr}}

library(tidyr) library(dplyr)

\hypertarget{we-are-not-getting-accurate-interval-level-summary-because}{%
\section{We are not getting accurate interval level summary
because}\label{we-are-not-getting-accurate-interval-level-summary-because}}

\#R see it as a factor
class(officew\_women3\(weight) class(officew_men3\)weight)

\hypertarget{let-us-change-it-to-numeric-or-interval}{%
\section{Let us change it to numeric or
interval}\label{let-us-change-it-to-numeric-or-interval}}

officew\_women3\(weight <-  as.numeric(as.character(officew_women3\)weight))

officew\_men3\(weight <-  as.numeric(as.character(officew_men3\)weight))

\hypertarget{let-us-check-us-again-great}{%
\section{Let us check us again,
great!}\label{let-us-check-us-again-great}}

class(officew\_women3\(weight) class(officew_men3\)weight)

\hypertarget{we-can-now-compare-the-means-of-gender}{%
\section{We can now compare the means of
gender:}\label{we-can-now-compare-the-means-of-gender}}

summary(officew\_women3\(weight) summary(officew_men3\)weight)

\hypertarget{sampling-inference-with-t-and-z-test}{%
\section{Sampling Inference with t and Z
test}\label{sampling-inference-with-t-and-z-test}}

\#What is the probabilty that a random male staff will weigh above 125.3

rand\_mstaff \textless- (125.3118 -
mean(officew\_men3\(weight))/sd(officew_men3\)weight)
1-pnorm(rand\_mstaff) 1- pnorm(2)

\#What is the probabilty that a random male staff will be \# obsess
(weigh above 100) \# It is 2.3 percent

rand\_menObestaff \textless- (100 -
mean(officew\_men3\(weight))/sd(officew_men3\)weight)
1-pnorm(rand\_menObestaff)

\#What is the probabilty that a random female staff will be \#
overweighted (weigh above 100 is overweight) \# It is 0.05 percent (less
than 1\%)

rand\_womenObestaff \textless- (100 -
mean(officew\_women3\(weight))/  sd(officew_women3\)weight)

1-pnorm(rand\_womenObestaff)

\hypertarget{hypothesis-testing}{%
\section{HYPOTHESIS TESTING}\label{hypothesis-testing}}

\#We know that the probability that the company will hire men with \#
with obseity is 2.3\%, and for women is about 1\%. It is clear that \#
the company hires more men with obesity than woman. However, what we \#
do not know if this difference is due to error in our random sampling \#
or truly reflect the differences in the entire staff. \# We are going to
use HYPOTHESIS TESTING by assuming first, that the \# difference between
them is zero referred to as NULL

\#Method 1 and 2: P-value \& T-test \# Do the t-test

t.test(officew\_women3\(weight, officew_men3\)weight)

\hypertarget{alternative-check-of-the-t-statistic}{%
\section{Alternative check of the
t-statistic:}\label{alternative-check-of-the-t-statistic}}

\hypertarget{this-is-similar-to-what-we-did-earlier-with-sampling-inference}{%
\section{This is similar to what we did earlier with sampling
inference}\label{this-is-similar-to-what-we-did-earlier-with-sampling-inference}}

\hypertarget{here-the-mean-is-zero-and-the-figure-weight-we-want-to-get-the}{%
\section{HERE, the mean is Zero, and the figure (Weight) we want to get
the}\label{here-the-mean-is-zero-and-the-figure-weight-we-want-to-get-the}}

\hypertarget{probabilty-for-is-the-the-mean-difference.-we-want-to-know-the}{%
\section{probabilty for is the the mean difference. We want to know
the}\label{probabilty-for-is-the-the-mean-difference.-we-want-to-know-the}}

\hypertarget{probability-of-the-mean-difference.-h1-is-the-mean-difference.}{%
\section{probability of the mean difference. H1 is the mean
difference.}\label{probability-of-the-mean-difference.-h1-is-the-mean-difference.}}

\hypertarget{h0-is-the-null}{%
\section{H0 is the NULL}\label{h0-is-the-null}}

\#6 steps to understanding the P-value
\#\#\#\#\#\#\#\#\#\#\#\#\#\#\#\#\#

\#1 We want to test the probability that makes us believe that an
effect(-15) \# or difference between two groups is NOT happening by
random chance

\#2.We assume the mean effect between these \# groups is zero meaning we
assume that there is no effect

\#3. We run a ttest to get the prob of that effect happening and check
the \# equivalent Probabiltiy. Note that we still assume our mean
difference is zero

\#4.We can say that if that probability we get is 5\% or less, then that
is \# probability of that effect occuring at an assumption of a mean of
zero.

\#5. It means the probability of that effect happening if we set our \#
mean difference to zero is very low. Since the probability that it will
occur is \#is very low, we should not accept that our mean difference is
zero.

\#6. So we reject that our mean difference is zero, and take the
alternative hypothesis.

\hypertarget{remember-that-if-the-t-value-is-2-or-1.96-or-more-it-means}{%
\section{Remember that if the T-value is 2 (or 1.96) or more, it
means}\label{remember-that-if-the-t-value-is-2-or-1.96-or-more-it-means}}

\hypertarget{the-probability-is-2.2-and-5-two-tail-or-less.-this-means-we-have}{%
\section{the probability is 2.2\% and 5\% (two-tail) or less. This means
we
have}\label{the-probability-is-2.2-and-5-two-tail-or-less.-this-means-we-have}}

\hypertarget{or-less-probabilty-that-we-will-by-chance-get-the-difference}{%
\section{5\% or less probabilty, that we will by chance get the
difference}\label{or-less-probabilty-that-we-will-by-chance-get-the-difference}}

\hypertarget{in-mean--15.1--the-effect-with-the-assumption-that-the-null}{%
\section{in mean (-15.1 -the effect) with the assumption that the
null}\label{in-mean--15.1--the-effect-with-the-assumption-that-the-null}}

\hypertarget{hypothesis-is-true-mean-is-zero}{%
\section{hypothesis is true (mean is
zero)}\label{hypothesis-is-true-mean-is-zero}}

\hypertarget{we-are-assuming-that-there-is-no-difference-but-that-assumption}{%
\section{We are assuming that there is no difference, but that
assumption}\label{we-are-assuming-that-there-is-no-difference-but-that-assumption}}

\hypertarget{will-only-occur-5-or-less-if-the-difference-is--15.1.-so-due-to-this}{%
\section{will only occur 5\% or less if the difference is -15.1. So due
to
this,}\label{will-only-occur-5-or-less-if-the-difference-is--15.1.-so-due-to-this}}

\hypertarget{we-reject-the-null-hypothesis-and-accept-the-alternative}{%
\section{we reject the null hypothesis and accept the
alternative}\label{we-reject-the-null-hypothesis-and-accept-the-alternative}}

\hypertarget{method-1}{%
\section{Method 1}\label{method-1}}

\hypertarget{first-calculate-standard-errors-for-both-groups}{%
\section{First: Calculate Standard Errors for both
groups}\label{first-calculate-standard-errors-for-both-groups}}

se.women \textless-
sd(officew\_women3\(weight) / sqrt(length(officew_women3\)weight))

se.men \textless-
sd(officew\_men3\(weight) / sqrt(length(officew_men3\)weight))

\hypertarget{sum-standardized-version-of-both-standard-errors}{%
\section{sum (standardized version of) both standard
errors:}\label{sum-standardized-version-of-both-standard-errors}}

se.diff \textless- sqrt((se.women\^{}2 + se.men\^{}2)) se.diff \# then
calculate confidence intervals: \# t = (H1 - H0) / sem.diff mean.diff
\textless- mean(officew\_women3\(weight) - mean(officew_men3\)weight)
mean.diff t \textless- (mean.diff - mean(0)) / se.diff t \# bigger than
1.96?

\hypertarget{calculating-t-value-at-95-confidence-interval-and-18-degree-of-freedom}{%
\section{calculating t-value at 95\% confidence interval and 18 degree
of
freedom}\label{calculating-t-value-at-95-confidence-interval-and-18-degree-of-freedom}}

qt(0.975, 18) \# I use a critical t value for 0.05 significance and 18
degrees of freedom df \textless-
nrow(officew\_women3\(weight) + nrow(officew_men3\)weight) - 2 df

\hypertarget{the-confidence-interval}{%
\section{the confidence interval}\label{the-confidence-interval}}

(qt(0.975, 18) * se.diff)

upper.ci \textless- mean.diff + (qt(0.975, 18) * se.diff)

lower.ci \textless- mean.diff - (qt(0.975, 18) * se.diff)

lower.ci upper.ci
\#\#\#\#\#\#\#\#\#\#\#\#\#\#\#\#\#\#\#\#\#\#\#\#\#\#\#\#\#\#\#\#\#\#\#\#\#\#\#\#\#\#
\# Confidence Interval and Replication by hand
\#\#\#\#\#\#\#\#\#\#\#\#\#\#\#\#\#\#\#\#\#\#\#\#\#\#\#\#\#\#\#\#\#\#\#\#\#\#\#\#\#\#

\#\#\#\#\#\#\#\#\#\#\#\#quiz example

weight\_men \textless- c(89, 75, 88, 75, 49, 89, 110, 120, 89, 75)

weight\_women \textless- c(75, 76, 87, 110, 67, 76, 43, 55, 59, 60)

\#method 1

t.test(weight\_men,weight\_women)

\#method 2 (by hand)

se.quiz.women \textless- sd(weight\_women) / sqrt(length(weight\_women
)) se.quiz.women

se.quiz.men \textless- sd(weight\_men) / sqrt(length(weight\_men ))
se.quiz.men

mean(weight\_men ) sd(weight\_men)

mean(weight\_women ) sd(weight\_women)

se.diff\_quiz \textless- sqrt((se.quiz.men\^{}2 + se.quiz.women\^{}2 ))
se.diff\_quiz

mean.diff55 \textless- mean(weight\_men) - mean(weight\_women )
mean.diff55

upper.ci\_quiz \textless- mean.diff55 + (qt(0.975, 18) * se.diff\_quiz)
upper.ci\_quiz

lower.ci\_quiz \textless- mean.diff55 - (qt(0.975, 18) * se.diff\_quiz)
lower.ci\_quiz

\#tvalue tvaluequiz \textless- (mean.diff55 - mean(0)) / se.diff\_quiz
\#t value tvaluequiz

?qt

obess\_quizm\textless- (100 - mean(weight\_men))/sd(weight\_men)
1-pnorm(obess\_quiz)

obess\_quizf\textless- (100 - mean(weight\_women))/sd(weight\_women)
1-pnorm(obess\_quizf)

p \textless- 0.22 n \textless- 1200 se.prop\textless-sqrt(p*(1-p))/sqrt
(n)

upperCI.prop \textless- p +(qt(0.975, (n-1))*se.prop) upperCI.prop

lowerCI.prop \textless- p -(qt(0.975, (n-1))*se.prop) lowerCI.prop

p1 \textless- 0.22 n1 \textless- 200
se.prop1\textless-sqrt(p1*(1-p1))/sqrt (n1)

upperCI.prop1 \textless- p1 +(qt(0.975, (n1-1))*se.prop1) upperCI.prop1

lowerCI.prop1 \textless- p1 -(qt(0.975, (n1-1))*se.prop1) lowerCI.prop1
\#\#quiz example ends

\hypertarget{interpretation-can-we-reject-h0}{%
\section{Interpretation? Can we reject
H0?}\label{interpretation-can-we-reject-h0}}

\hypertarget{no-we-cannot-reject-the-null-hypothesis-that-the-difference-in-the-mean}{%
\section{No, we cannot reject the null hypothesis that the difference in
the
mean}\label{no-we-cannot-reject-the-null-hypothesis-that-the-difference-in-the-mean}}

\hypertarget{of-both-gender-is-zero-at-95-confidence.-we-are-95-confident-that-the}{%
\section{of both gender is zero at 95\% confidence. We are 95\%
confident that
the}\label{of-both-gender-is-zero-at-95-confidence.-we-are-95-confident-that-the}}

\hypertarget{difference-in-mean-of-both-gender-is-zero}{%
\section{difference in mean of both gender is
zero}\label{difference-in-mean-of-both-gender-is-zero}}

\hypertarget{we-can-also-say-that-we-cannot-reject-the-null-hypothesis-that-the-difference-in-the-mean}{%
\section{We can also say that we cannot reject the null hypothesis that
the difference in the
mean}\label{we-can-also-say-that-we-cannot-reject-the-null-hypothesis-that-the-difference-in-the-mean}}

\hypertarget{in-mean-of-both-gender-will-happen-by-random-chance-5-times-or-more}{%
\section{in mean of both gender will happen by random chance 5 times or
more}\label{in-mean-of-both-gender-will-happen-by-random-chance-5-times-or-more}}

\hypertarget{out-of-every-100-occurence.}{%
\section{out of every 100
occurence.}\label{out-of-every-100-occurence.}}

\#Using the First Method (t-value test), the t-statistic is -1.96, and
\# our t-value is -1.75. If we plot this in a graph, -1.75 falls within
regions \# lower than -1.96 (0.05 p-value), but we were only ready to
accept \#region at -1.96 and above it

\#Using the second method (p-value test), our p-value as well is higher
than 0.05. Our P-value \#is the probability of getting a result as
extreme as our test statistic, \#assuming our NULL hypothesis is true
that there is no difference in the mean.

\#Using the third method (CI test), our confidence interval is -33 to
2.with \# the mean as -15.1. Zero (0)is within the confidence interval
that we are \# are 95 \% confident the difference in mean between both
gender can also be 0. \# We cannot reject the null hypothesis in this
regard

\hypertarget{another-example-of-t-test-with-normal-distribution}{%
\section{ANOTHER EXAMPLE OF T-TEST WITH NORMAL
DISTRIBUTION}\label{another-example-of-t-test-with-normal-distribution}}

\hypertarget{make-simulations-replicable}{%
\section{make simulations
replicable:}\label{make-simulations-replicable}}

set.seed(101112) \# disable scientific notation: options(scipen=999)

\hypertarget{we-start-by-creating-two-different-normally-distributed-variables}{%
\section{We start by creating two different normally distributed
variables:}\label{we-start-by-creating-two-different-normally-distributed-variables}}

var1 \textless- rnorm(50, mean = 0, sd = 1) var2 \textless- rnorm(100,
mean = 0.5, sd = 3)

\hypertarget{what-are-their-means}{%
\section{what are their means?}\label{what-are-their-means}}

mean(var1) mean(var2)

\hypertarget{is-there-a-significant-difference}{%
\section{is there a significant
difference?}\label{is-there-a-significant-difference}}

t.test(var1, var2)

\hypertarget{chi-square}{%
\section{Chi-square}\label{chi-square}}

\#Chi-Square test in R is a statistical method \#which used to determine
if two categorical variables have a \#significant correlation between
them

\#The difference with x2 is between the observed frequency (fo) and
\#the expected frequency (fe).

\#H0: every i.v. category should have the \#same distribution across the
d.v. as the total, \#i.e.~i.v. doesn't matter.

\#Let us assume you that in the process of the review, \# an arguement
from one of your HR staff is that men \# are more single than women in
the organization.

\hypertarget{we-are-interested-in-knowing-whether-gender-affect-being-single}{%
\section{We are interested in knowing whether gender affect being
single}\label{we-are-interested-in-knowing-whether-gender-affect-being-single}}

\hypertarget{we-want-to-know-if-whether-either-you-are-a-male-of-female}{%
\section{We want to know if whether either you are a male of
female}\label{we-want-to-know-if-whether-either-you-are-a-male-of-female}}

\#has an effect on the marital status

\#Here gender is the IV and Martital Status is the DV

\hypertarget{step-1--derive-a-contigency-table}{%
\section{STEP 1- DERIVE A CONTIGENCY
TABLE}\label{step-1--derive-a-contigency-table}}

\#Let us first divide our martital status into two concrete divisions -
\#Single Vs Not Single
table(officew\(Marstatus) table(officew\)Marstatus==``Single'')
officew\(NewMarStatus<- ifelse(officew\)Marstatus==``Single'',
``Single'', ``Not Single'')

\hypertarget{we-then-use-the-table-function-to-show-the-cross-tab}{%
\section{We then use the table function to show the cross
tab}\label{we-then-use-the-table-function-to-show-the-cross-tab}}

\#Converting NewMarStatus to a Factor
officew\(NewMarStatus <- as.factor(as.character(officew\)NewMarStatus))

table(officew\(Gender, officew\)NewMarStatus) \# To get the percentages,
we use prop.table function

prop.table(table( officew\(NewMarStatus,officew\)Gender), 2) \#\#\# 50
percent of female are single, \#\#\# and 60 percent of males are single.
\#\#\# We can say the effect of being a male is 10 percentage point
higher \# for men than women.
prop.table(table(officew\(NewMarStatus, officew\)Gender), 1 )

\#ALTERNATIVE 2: Install Gmodel package install.packages(``gmodels'')
library(gmodels)

CrossTable(officew\(Gender, officew\)NewMarStatus)

\hypertarget{let-us-assume-we-can-to-control-for-location.-we-think}{%
\section{Let us assume we can to CONTROL for location. We
think}\label{let-us-assume-we-can-to-control-for-location.-we-think}}

\hypertarget{we-can-also-use-the-location-to-staff-either-they-stay-in-the-central-city}{%
\section{we can also use the location to staff (either they stay in the
central
city)}\label{we-can-also-use-the-location-to-staff-either-they-stay-in-the-central-city}}

across both genders to know wnether they are single or not

\hypertarget{you-just-need-to-insert-the-new-variable-to-the-table-function}{%
\section{You just need to insert the new variable to the TABLE
function}\label{you-just-need-to-insert-the-new-variable-to-the-table-function}}

?table table(officew\(Gender, officew\)NewMarStatus,
officew\(CityCentral) prop.table (table(officew\)Gender,
officew\(NewMarStatus, officew\)CityCentral), 3)

\#it might be more convenient to create two subsets of the data \# one
for those who live in Central Area, and one for those \# who don't

\hypertarget{for-people-who-live-in-the-central-area}{%
\section{For people who live in the central
area}\label{for-people-who-live-in-the-central-area}}

officew\_central \textless- officew{[}officew\$CityCentral==``Yes'',{]}
\#For people who DO NOT live in the central area officew\_Nocentral
\textless- officew{[}officew\$CityCentral==``No'',{]}

\#With these subsets, you can obtain the cross-tabulations separately
and \# in percentage form

\hypertarget{for-people-who-live-in-the-central-area-1}{%
\section{For people who live in the central
area}\label{for-people-who-live-in-the-central-area-1}}

prop.table
(table(officew\_central\(Gender, officew_central\)NewMarStatus),2)

\hypertarget{for-people-who-do-not-live-in-the-central-area}{%
\section{For people who DO NOT live in the central
area}\label{for-people-who-do-not-live-in-the-central-area}}

prop.table
(table(officew\_Nocentral\(Gender, officew_Nocentral\)NewMarStatus),2)

\#STE 2: CONDUCT a t-test and check the chi square(x2) and p value

chisq.test(officew\(Gender,officew\)NewMarStatus,correct=FALSE)

\hypertarget{it-gave-the-warning-because-many-of-the-expected-values-will-be-very-small}{%
\subsection{It gave the warning because many of the expected values will
be very
small}\label{it-gave-the-warning-because-many-of-the-expected-values-will-be-very-small}}

\hypertarget{and-therefore-the-approximations-of-p-may-not-be-right.}{%
\subsection{and therefore the approximations of p may not be
right.}\label{and-therefore-the-approximations-of-p-may-not-be-right.}}

\#\#In R you can use chisq.test(a, simulate.p.value = TRUE) to use
\#\#simulate p values.

chisq.test(officew\(Gender,officew\)NewMarStatus, simulate.p.value =
TRUE)

\hypertarget{however-with-such-small-cell-sizes-all-estimates-will-be-poor.}{%
\subsection{However, with such small cell sizes, all estimates will be
poor.}\label{however-with-such-small-cell-sizes-all-estimates-will-be-poor.}}

\#\#It might be good to just test pass vs.~fail (deleting ``no show'')
\#\#either with chi-square or logistic regression. Indeed, \#\#since it
is pretty clear that the pass/fail grade is a dependent variable,
\#\#logistic regression might be better

\hypertarget{question-at-what-significance-level-will-our-chi-square-statistic-of-0.20}{%
\subsection{QUESTION: At what significance level will our chi square
statistic of
0.20}\label{question-at-what-significance-level-will-our-chi-square-statistic-of-0.20}}

\hypertarget{at-degree-of-freedom.-please-look-at-chisquare-table}{%
\subsection{at ----- degree of freedom. PLease look at Chisquare
table}\label{at-degree-of-freedom.-please-look-at-chisquare-table}}

\hypertarget{be-significant}{%
\subsection{be significant}\label{be-significant}}

\hypertarget{correlation}{%
\section{Correlation}\label{correlation}}

\hypertarget{packages}{%
\subsection{packages:}\label{packages}}

install.packages(``corrplot'') \# Install the corrplot library, for
nice-looking \# correlation plots. Do this once.
install.packages(``ggplot2'') \# Install the ggplot2 package, for
high-quality \# graphs install.packages(``cowplot'') \# Install the
cowplot package, to arrange plots \# into a grid
install.packages(``ggpubr'') \#\#\#\#

\hypertarget{optional-packages}{%
\paragraph{Optional packages:}\label{optional-packages}}

library(corrplot) \# Plotting nice correlation matrix library(cowplot)
\# arranging plots into a grid library(ggplot2) \# high-quality graphs

\hypertarget{section-19}{%
\paragraph{}\label{section-19}}

str(officew)\#\#\#\#check the data properties ?cor cor(officew) \#you
get an error -\textgreater{} variables should be numeric.

\hypertarget{let-us-change-it-to-numeric-or-interval-1}{%
\section{Let us change it to numeric or
interval}\label{let-us-change-it-to-numeric-or-interval-1}}

officew\(weight <-as.numeric(as.character(officew\)weight))

officew\(income <- as.numeric(as.character(officew\)income))

officew\(rating <- as.numeric(as.character(officew\)rating))

\hypertarget{let-us-check-us-again-great-1}{%
\section{Let us check us again,
great!}\label{let-us-check-us-again-great-1}}

class(officew\(weight) class(officew\)income) class(officew\$rating)

\hypertarget{plot-the-graph-use-y-as-income}{%
\section{plot the graph , use y as
income}\label{plot-the-graph-use-y-as-income}}

library(``ggpubr'') ggscatter(officew, x = ``rating'', y = ``weight'',
add = ``reg.line'', conf.int = TRUE, cor.coef = TRUE, cor.method =
``pearson'', xlab = ``Rating in the sample'', ylab = ``Weight (in Kg)'')

cor(officew\(weight,officew\)rating)

?cor cor(officew) \#you get an error -\textgreater{} variables should be
numeric. Why?? corr1\textless- cor(officew{[}c(-1,-5,-6,-7){]}) \#we do
it without the \#last non-numeric variable ``type'' which are indexied
in 3, 4 and 5 corr1

\#nice correlation matrix ?corrplot corrplot(corr1, method = ``number'')
\# Try with different methods!

\hypertarget{regression}{%
\section{REGRESSION}\label{regression}}

\hypertarget{step-1-check-the-plot}{%
\section{STEP 1, CHECK THE PLOT}\label{step-1-check-the-plot}}

\hypertarget{how-does-this-look-lets-plot-the-data}{%
\section{How does this look? Let's plot the
data!}\label{how-does-this-look-lets-plot-the-data}}

plot(officew\(weight, officew\)income) \# First x axis, then y axis: \#
our independent variable is income

\hypertarget{step-2-run-your-first-bivariate-regression}{%
\section{STEP 2 Run your first (bivariate)
regression}\label{step-2-run-your-first-bivariate-regression}}

myfirstreg \textless- lm(weight \textasciitilde{} income, data= officew)
\#First you run it summary(myfirstreg) \#Then you see the output

officew\(weight <- as.numeric(as.character(officew\)weight))

class(officew\$weight)

\hypertarget{if-you-would-like-to-take-a-look-at-the-confidence-intervals-at}{%
\section{If you would like to take a look at the confidence intervals
at}\label{if-you-would-like-to-take-a-look-at-the-confidence-intervals-at}}

\hypertarget{confidence-level-of-99}\label{confidence-level-of-99}}

confint(myfirstreg, level=0.99)

\#STEP 3 PRINT your Result
\#\#\#\#\#\#\#\#\#\#\#\#\#\#\#\#\#\#\#\#\#\#\#\#\#\#\#\#\#\#\#\#\#
install.packages(``stargazer'') \# Install stargazer, for nice-looking
regression \# tables. Do this once

library(stargazer) \# Lets get a nice table out of it
stargazer(myfirstreg, title=``Regression Results'', out=``reg.txt'')

\#STEP 4 Interprete the result
\#\#\#\#\#\#\#\#\#\#\#\#\#\#\#\#\#\#\#\#\#\#\#\#\#\#\#\#\#\#\#\#\#

options(scipen=999) \#run this once to turn off scientific notation in
\#your reg output myfirstreg2 \textless- lm(weight \textasciitilde{}
income+rating, data= officew) \#First you run it summary(myfirstreg2)
\#Then you see the output

\hypertarget{multicollonearity-vif-tolerance-post-treatment-effect}{%
\section{MULTICOLLONEARITY (VIF \& TOLERANCE) \& Post Treatment
Effect}\label{multicollonearity-vif-tolerance-post-treatment-effect}}

install.packages(``carData'') install.packages(``car'') library(car)
library(carData)

\hypertarget{vif-and-tolerance}{%
\section{VIF and Tolerance}\label{vif-and-tolerance}}

\hypertarget{let-us-test-for-the-multicollinearity-of-both-rating-and-income}{%
\section{Let us test for the multicollinearity of both rating and
income}\label{let-us-test-for-the-multicollinearity-of-both-rating-and-income}}

\hypertarget{on-our-dependent-variable}{%
\section{on our dependent variable}\label{on-our-dependent-variable}}

\hypertarget{vif-test}{%
\section{VIF test}\label{vif-test}}

\hypertarget{the-square-root-of-the-vif-tells-us-by-which-factor-at-which-the}{%
\section{The square root of the VIF tells us by which factor at which
the}\label{the-square-root-of-the-vif-tells-us-by-which-factor-at-which-the}}

\hypertarget{standard-error-for-the-coefficient-of-the-iv-will-be-larger-than-if}{%
\section{standard error for the coefficient of the IV will be larger
than
if}\label{standard-error-for-the-coefficient-of-the-iv-will-be-larger-than-if}}

\hypertarget{that-if-it-had-0-correlation-with-other-independent-variables.}{%
\section{that if it had 0 correlation with other independent
variables.}\label{that-if-it-had-0-correlation-with-other-independent-variables.}}

?vif

vif(myfirstreg2) \# The Standard Error of the CE of income will be
inflated by 1.012 if we \# include it in the model.

\#Tolerance is proportion of the model's independent variables not \#
explained by other independent variables

\hypertarget{the-tolerance-is-the-inverse-of-the-vif-and-is-the-percent-of-variance}{%
\section{The tolerance is the inverse of the vif and is the Percent of
variance}\label{the-tolerance-is-the-inverse-of-the-vif-and-is-the-percent-of-variance}}

\hypertarget{in-the-predictor-that-cannot-be-accounted-for-by-other-predictors.}{%
\section{in the predictor that cannot be accounted for by other
predictors.}\label{in-the-predictor-that-cannot-be-accounted-for-by-other-predictors.}}

1/vif(myfirstreg2) \# For example, if you run the VIF, 98 percent of the
variance of income \# cannot be explained by other. This is where there
is no \# correlation. It is what is unique to this variable income, that
can't be \# explained by any other in the set

\hypertarget{post-treatment-bias}{%
\section{Post Treatment Bias}\label{post-treatment-bias}}

install.packages(AER) library(AER) ?Fatalities data(``Fatalities'')

fatal \textless- lm(fatal\textasciitilde beertax + youngdrivers + miles
+ pop, data = Fatalities) summary(fatal) \# model with a number of
covariates to isolate effect of drunk driving

fatal.ptb \textless- lm(fatal \textasciitilde{} beertax +youngdrivers +
miles + pop + spirits, data = Fatalities) summary(fatal.ptb) \# adding
control for the mechanism (spirits consumption)

stargazer(fatal, fatal.ptb, type = ``text'', data = Fatalities, out =
``fatal.reg.txt'')

\hypertarget{here-controling-for-the-mechanism-causes-part-of-the-effect-of-beertax-to}{%
\section{Here controling for the mechanism causes part of the effect of
beertax
to}\label{here-controling-for-the-mechanism-causes-part-of-the-effect-of-beertax-to}}

\hypertarget{be-mathematically-soaked-up.-admittedly-the-effect-is-a-little-weak.}{%
\section{be mathematically ``soaked up''. Admittedly, the effect is a
little
weak.}\label{be-mathematically-soaked-up.-admittedly-the-effect-is-a-little-weak.}}

\hypertarget{dummy-variable}{%
\section{Dummy Variable}\label{dummy-variable}}

\hypertarget{convert-gender-to-dummy-variable-where-male-is-1}{%
\section{Convert gender to dummy variable, where male is
1}\label{convert-gender-to-dummy-variable-where-male-is-1}}

\#and female is 0. Male is our baseline variable

officew\(dMale<- ifelse(officew\)Gender==``Male'', 1, 0)

\hypertarget{run-regression}{%
\section{Run regression}\label{run-regression}}

reg11 \textless- lm(weight \textasciitilde{} dMale,data = officew)
summary (reg11)

\#INTEPRETE THE RESULT \#You interprete the male coefficient to the
other \# result to the Other dummy variable and not the whole model \#
Men on average, will have an addition increase in Kg by 15 units than
women

\hypertarget{run-regression-1}{%
\section{Run regression}\label{run-regression-1}}

reg10 \textless- lm(income \textasciitilde{} dMale,data = officew)
summary(reg10)

\hypertarget{does-income-have-any-effect-hardly-significant-on-the-model}{%
\section{Does income have any effect, hardly significant on the
model}\label{does-income-have-any-effect-hardly-significant-on-the-model}}

options(scipen=999) \#run this once to turn off scientific notation in
\#your reg output

\hypertarget{run-regression-again}{%
\section{Run regression again}\label{run-regression-again}}

reg12 \textless- lm(weight \textasciitilde{} dMale + income,data =
officew) summary (reg13)

\hypertarget{ggpplot-and-interaction-effect}{%
\section{GGPPLOT and INTERACTION
EFFECT}\label{ggpplot-and-interaction-effect}}

\hypertarget{graphically-using-ggplot}{%
\section{graphically using ggplot}\label{graphically-using-ggplot}}

ggplot(officew, aes(x = weight, y = income, colour = rating)) +
geom\_point() + geom\_smooth(method = ``lm'')

\hypertarget{what-if-we-are-interested-on-if-the-effect-of-income-on-weight-is-different}{%
\section{WHat if we are interested on if the effect of income on weight
is
different}\label{what-if-we-are-interested-on-if-the-effect-of-income-on-weight-is-different}}

\hypertarget{across-gender}{%
\section{across gender}\label{across-gender}}

reg12 \textless- lm(weight \textasciitilde{} dMale + income,data =
officew) summary (reg12)

reg15 \textless- lm(weight \textasciitilde{} dMale + income +
dMale*income,data = officew) summary (reg15)

reg18 \textless- lm(weight \textasciitilde{} income+ , data = officew)
summary (reg18)

\begin{Shaded}
\begin{Highlighting}[]
\FunctionTok{summary}\NormalTok{(cars)}
\end{Highlighting}
\end{Shaded}

\begin{verbatim}
##      speed           dist       
##  Min.   : 4.0   Min.   :  2.00  
##  1st Qu.:12.0   1st Qu.: 26.00  
##  Median :15.0   Median : 36.00  
##  Mean   :15.4   Mean   : 42.98  
##  3rd Qu.:19.0   3rd Qu.: 56.00  
##  Max.   :25.0   Max.   :120.00
\end{verbatim}

\hypertarget{including-plots}{%
\subsection{Including Plots}\label{including-plots}}

You can also embed plots, for example:

\includegraphics{FLS301_2021_files/figure-latex/pressure-1.pdf}

Note that the \texttt{echo\ =\ FALSE} parameter was added to the code
chunk to prevent printing of the R code that generated the plot.

\end{document}
